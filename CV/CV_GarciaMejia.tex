%%%%%%%%%%%%%%%%%%%%%%%%%%%%%%%%%%%%%%%%%
%Based on: 
% Plasmati Graduate CV by Alessandro Plasmati (alessandro.plasmati@gmail.com)
% LaTeX Template
% Version 1.0 (24/3/13)
% License:
% CC BY-NC-SA 3.0 (http://creativecommons.org/licenses/by-nc-sa/3.0/)
%
% Important note:
% This template needs to be compiled with XeLaTeX.

%%%%%%%%%%%%%%%%%%%%%%%%%%%%%%%%%%%%%%%%%

%----------------------------------------------------------------------------------------
%	PACKAGES AND OTHER DOCUMENT CONFIGURATIONS
%----------------------------------------------------------------------------------------

\documentclass[a4paper,11pt]{article} % Default font size and paper size

\usepackage[T1]{fontenc}
\usepackage{helvet}

%\usepackage{fontspec} % For loading fonts
%\defaultfontfeatures{Mapping=tex-text}
%\setmainfont{Helvetica} % Main document font

%\usepackage[style=authoryear,backend=biber]{biblatex} %
\usepackage{bookmark}

\usepackage{xunicode,xltxtra,url,parskip} % Formatting packages

\usepackage{tabularx} %This allows making tables where long lines get automatically broken.

\usepackage{longtable} %This allows to create tables that break between pages

\usepackage{array}

\usepackage{enumitem} %To set length of items

\usepackage[usenames,dvipsnames]{xcolor} % Required for specifying custom colors

\usepackage[big]{layaureo} % Margin formatting of the A4 page, an alternative to layaureo can be \usepackage{fullpage}
% To reduce the height of the top margin uncomment: \addtolength{\voffset}{-1.3cm}

\usepackage{hyperref} % Required for adding links and customizing them
\definecolor{linkcolour}{rgb}{0.6,0.1,0.2} % Link color
\hypersetup{colorlinks,breaklinks,urlcolor=linkcolour,linkcolor=linkcolour} % Set link colors throughout the document

\usepackage{titlesec} % Used to customize the \section command
\titleformat{\section}{\large\scshape\raggedright}{}{0em}{}[\titlerule] % Text formatting of sections
\titlespacing{\section}{0pt}{8pt}{8pt} % Spacing around sections

\begin{document}

\pagestyle{empty} % Removes page numbering

%----------------------------------------------------------------------------------------
%	NAME AND CONTACT INFORMATION
%----------------------------------------------------------------------------------------

\par{\centering{\LARGE Jer\'onimo Garc\'ia-Mej\'ia}\bigskip\par} % Your name

\section{Personal Data}

\begin{longtable}{>{\raggedleft\arraybackslash}p{4cm}p{10cm}}
%\textsc{Address:} & University of Oxford, Mathematical Institute \\
%\textsc{Phone:} & +1 111 1112\\
\textsc{Email:} & \href{mailto:jeronimo.garcia-mejia@maths.ox.ac.uk}{jeronimo.garcia-mejia@maths.ox.ac.uk}\\
\textsc{Website:} & \href{https://jeronimomaths.github.io}{jeronimomaths.github.io} \\
\textsc{Research Interests:} & Geometric group theory. In particular, filling functions of groups and spaces, with a focus on Dehn functions, nilpotent groups and their QI-classification, and automorphism groups of non-positively curved groups.\end{longtable}
%----------------------------------------------------------------------------------------
%	WORK EXPERIENCE 
%----------------------------------------------------------------------------------------

\section{Academic Positions}

\begin{longtable}{>{\raggedleft\arraybackslash}p{4cm}p{10cm}}
\textsc{Dec. 2024 -- Present} & \textsc{University of Oxford} \vspace{0.2em} \\
& Postdoctoral Research Associate \\
& Mathematical Institute \\
& \quad Postdoctoral Mentor: Ric Wade
\end{longtable}

%----------------------------------------------------------------------------------------
%	EDUCATION
%----------------------------------------------------------------------------------------

\section{Education}

\begin{longtable}{>{\raggedleft\arraybackslash}p{4cm}p{10cm}}
\textsc{Sept.} 2024 & \textsc{Karlsruhe Institute of Technology} \vspace{0.3em} \\
& PhD in Mathematics, Grade: Magna Cum Laude \vspace{0.3em}\\
& \quad Thesis: \emph{Dehn functions of groups with filiform subgroups}\\ 
& \quad Advisor: Jun.-Prof. Claudio Llosa Isenrich \vspace{0.5em}\\

\textsc{Aug.} 2020& \textsc{University of Bonn} \vspace{0.3em}\\
&M.Sc. in Mathematics, Grade: 1.8 \vspace{0.3em}\\
& \quad Thesis: \emph{Group actions on hyperbolic spaces}\\ 
& \quad Advisor: Prof. Ursula Hamenstädt \vspace{0.5em}\\

\textsc{Apr.} 2018& \textsc{Universidad Nacional Autónoma de México} \vspace{0.3em}\\ 
& B.Sc. in Mathematics, Grade: 9.3 \vspace{0.3em}\\
& \quad Thesis: \emph{Construction of classifying spaces} (in Spanish)\\ 
& \quad Advisor: Prof. Carlos Prieto\\
\end{longtable}

%----------------------------------------------------------------------------------------
%	SCHOLARSHIPS AND ADDITIONAL INFO
%----------------------------------------------------------------------------------------

\section{Scolarships}

\begin{longtable}{>{\raggedleft\arraybackslash}p{4cm}p{10cm}}

\textsc{2022--2024} & \textsc{Deutsche Forschungsgemeinschaft} \small (German Research Foundation)\vspace{0.2em}\\
& PhD Scolarship \vspace{0.2em}\\

\textsc{2018--2020} & \textsc{Rosa Luxemburg Stiftung} \small (Rosa Luxemburg Foundation)\vspace{0.2em}\\
& Scolarship
\end{longtable}

%----------------------------------------------------------------------------------------
%	PUBLICATIONS
%----------------------------------------------------------------------------------------

\section{Publications}

\begin{minipage}{15cm}
\begin{enumerate}[align=right, itemsep=.5em, leftmargin=1.8em]
	
	\item Jerónimo García-Mejía and Antoine Goldsborough, \emph{Random Dehn function of groups}. To appear in {\href{https://doi.org/10.1142/S179352532550027X}{\textbf{Journal of Topology and Analysis}}} {\href{https://arxiv.org/abs/2411.12715}{(\textbf{arXiv})}}.
    
\end{enumerate}
\end{minipage}

\section{Preprints}

\begin{minipage}{15cm}
\begin{enumerate}[align=right, itemsep=.5em, leftmargin=1.8em]
	
	\item Yu-Chang Chan, Jerónimo García-Mejía, and Matteo Migliorini, \emph{Complete classification of the Dehn functions of Bestvina--Brady groups}, Submitted (Jul. 2025), {\href{https://arxiv.org/abs/2507.07566}{\textbf{arXiv:2507.07566}}}.
    
    \item Jerónimo García-Mejía, Claudio Llosa Isenrich and Gabriel Pallier \emph{On Dehn functions of central products of nilpotent groups}, Under review,  (Oct. 2023), {\href{https://arxiv.org/abs/2310.11144}{\textbf{arXiv:2310.11144}}}.
    
\end{enumerate}
\end{minipage}

\section{Other work containing original research}

\begin{itemize}
    \item \emph{Dehn functions of groups with filiform subgroups}. \\
    PhD thesis, Karlsruhe Institute of Technology. 

    \item \emph{Group actions on hyperbolic spaces}. \\
    Master thesis, University of Bonn. Unpublished. 
\end{itemize}

%----------------------------------------------------------------------------------------
%	TALKS SECTION
%----------------------------------------------------------------------------------------

\section{Invited talks}

%\begin{tabularx}{\textwidth}{rX} %The column type X takes all the space left from other columns till \textwidth. Tabularx does not allow page breaks.

\begin{longtable}{rp{0.8\linewidth}}
	
	\textsc{May 2025} & \emph{Dehn functions of Bestvina--Brady groups}, \href{https://talks.cam.ac.uk/talk/index/229831}{Geometric Group Theory Seminar}, University of Cambridge, Cambridge, United Kigndom. \\

    \textsc{July 2024} & \emph{Dehn functions of nilpotent groups}, \href{https://www.maths.ox.ac.uk/node/68709}{Geometric Group Theory Seminar}, University of Oxford, Oxford, United Kingdom. \\

    \textsc{Apr. 2024} & \emph{Dehn functions of nilpotent groups}, \href{https://www.uni-muenster.de/Logik/en/Forschung/Research_Seminar_Model_theory_groups.html}{Research seminar Model theory and groups}, University of Münster, Münster, Germany. \\

    \textsc{Mar. 2024} & \emph{Funciones de Dehn de grupos nilpotentes}, \href{https://sites.google.com/view/seminar-ggt-mexico/}{Long-distance seminar in Geometric Group Theory in Mexico}, Online Seminar.\\

    \textsc{Jan. 2024} & \emph{Dehn functions of central products of nilpotent groups}, \href{https://math.univ-lille.fr/agenda/seminaires/seminaire-geometrie-dynamique}{Geometry and Dynamics Seminar}, Université de Lille, Lille, France. \\

    \textsc{Oct. 2022} & \emph{The Geometry of the Word Problem}, \href{https://sites.google.com/view/babygeometri/english}{Baby Geometri Seminar}, University of Pisa - Scuola Normale Superiore, Pisa, Italy.\\

    \textsc{May 2020} & \emph{Group actions on hyperbolic spaces}, \href{https://www.mathi.uni-heidelberg.de/~arandecker/junior_seminar_SS20}{Junior Geometry Seminar}, Heidelberg University, Heidelberg, Germany.\\
\end{longtable}

\section{Contributed talks}

\begin{longtable}{rp{0.8\linewidth}}

    \textsc{May 2023} & \emph{Lightning Talk}, Conference Geometry of subgroups, Thematic program on Geometric Group Theory, Centre de Recherches Mathématiques, Montréal, Canada. \\

    \textsc{Feb. 2023} & \emph{Lightning Talk}, Young Geometric Group Theory XI, Münster, Germany.\\

%    \textsc{Dec. 2022} & \emph{Dehn functions of central products of nilpotent groups}, AG Seminar: Topology and Geometric Group Theory, Karlsruhe Institute of Technology, Karlsruhe, Germany. \\

%    \textsc{Jan. 2022} & \emph{Bestvina-Brady Morse theory}, GCD Seminar, Karlsruhe Institute of Technology, Karlsruhe, Germany. \\

    \textsc{Feb. 2021} & \emph{Group actions, hyperbolic spaces and small cancellation}, AG Seminar Metric Geometry, Karlsruhe Institute of Technology, Karlsruhe, Germany.\\
\end{longtable}

%----------------------------------------------------------------------------------------
%	CONFERENCES
%----------------------------------------------------------------------------------------

\section{Conferences and workshops attendance}

\begin{longtable}{rp{0.8\linewidth}}

	\textsc{Jul. 2025} &\emph{Actions on graphs and metric spaces} part of the Program Operators, Graphs, Groups, Isaac Newton Institute for Mathematical Sciences, Cambridge, United Kingdom. \\ % 01.09.2025 - 05.09.2025

	\textsc{Jul. 2025} &\emph{Non-positive curvature and applications}, Isaac Newton Institute for Mathematical Sciences, Cambridge, United Kingdom. \\ % 07.07.2025 - 18.07.2025
	
	\textsc{Jun. 2025} & \emph{The William Rowan Hamilton Geometry and Topology Workshop: Celebrating Martin Bridson's 60th Birthday}, The Hamilton Mathematics Institute, Trinity College Dublin, Dublin, Ireland. \\ % 30.06.2025 - 04.07.2025
	
	\textsc{May 2025} & \emph{Groups and Geometry in South England}, University of Oxford, Oxford, England. \\ % 30.05.2025

    \textsc{Aug. 2024} & \emph{Workshop on Profinite Rigidity}, Karlsruhe Institute of Technology, Karlsruhe,
    Germany. \\ % 26.08.2024 - 28.08.2024

    \textsc{Apr. 2024} & \emph{Young Geometric Group Theory XII}, University of Bristol, Bristol, United Kingdom. \\ % 08.04.2024 - 12.04.2024
    
    \textsc{Mar. 2024} & \emph{Topological and Homological Methods in Group Theory 2024}, Bielefeld
    University, Bielefeld, Germany. \\ % 18.03.2024 - 22.03.2024

    \textsc{Jun. 2023} & \emph{Cube complexes and combinatorial group theory} part of the Thematic program on Geometric group theory, Centre de Recherches Mathématiques, Montréal, Canada. \\ % 19.06.2023 - 30.06.2023
    
    \textsc{Jun. 2023} & \emph{Groups around 3-manifolds} part of the Thematic program on Geometric group theory,
    Centre de Recherches Mathématiques, Montréal, Canada. \\ %  05.06.2023 - 16.06.2023

    \textsc{May 2023} & \emph{Geometry of subgroups} part of the Thematic program on Geometric group theory, Centre
    de Recherches Mathématiques, Montréal, Canada. \\ % 15.05.2023 - 26.05.2023

    \textsc{Feb. 2023} & \emph{Young Geometric Group Theory XI}, Münster University, Münster, Germany. \\ % 13.02.2023 - 17.02.2023

    \textsc{Jul. 2022} & \emph{Metric Geometry and Geometric Analysis}, MSRI-Oxford Summer Graduate
    School, Oxford University, Oxford, United Kingdom. \\ % 11.07.2022 - 22.07.2022

    \textsc{Jun. 2022} & \emph{Hyperbolic groups and their generalisations}, Institut Henri Poincaré, Paris,
    France. \\ % 20.06.2022 - 24.06.2022

    \textsc{Mar. 2022} & \emph{Topological and Homological Methods in Group Theory 2022}, Bielefeld University, Bielefeld, Germany. \\ % 28.03.2022 - 01.04.2022
    
    \textsc{Feb. 2022} & \emph{Workshop on Finiteness properties}, Karlsruhe Institute of Technology, Karlsruhe, Germany. \\ % 23.02.2022 - 25.02.2022

\end{longtable}

%----------------------------------------------------------------------------------------
%	Extended visits
%----------------------------------------------------------------------------------------


\section{Extended visits}

\begin{longtable}{>{\raggedleft\arraybackslash}p{4cm}p{10cm}}

\textsc{Sep. 2025} &\textsc{Isaac Newton Institute} \\ &Program \emph{Operators, Graphs, Groups} \vspace{0.5em}\\

\textsc{May -- Jun. 2023} &\textsc{Centre de Recherches Mathématiques} \\ & Thematic program on \emph{Geometric group theory}

\end{longtable}

%----------------------------------------------------------------------------------------
%	TEACHING EXPERIENCE
%----------------------------------------------------------------------------------------

\section{Teaching experience}

\begin{minipage}{15cm}
My teaching experience as \emph{Teacher's Assistant} covers preparing exercise-sheets, presenting solutions in front of the class, and grading. All of this in Spanish, English, and German.
\end{minipage}

\begin{longtable}{>{\raggedleft\arraybackslash}p{4.2cm}p{10cm}}

    \textsc{Winter 2021/22} & \emph{Elementare Geometrie (Elementary Geometry)}, Bachelor course in German at the Karlsruhe Institute of Technology. \\

    \textsc{Summer 2021} & \emph{Geometric Group Theory}, Master course in English at the Karlsruhe Institute of Technology.\\

    \textsc{Winter 2020/21} & \emph{Proseminar Geometrie: Knotentheorie (Knot theory)}, Bachelor Seminar in German at the Karlsruhe Institute of Technology.\\

    \textsc{Winter 2017/18} & \emph{Analytical Geometry I} and \emph{Topology I}, Bachelor courses in Spanish at Universidad Nacional Autónoma de México.\\
    
\end{longtable}

%----------------------------------------------------------------------------------------
%	SERVICE SECTION
%----------------------------------------------------------------------------------------
	
\section{Services}

\begin{longtable}{>{\raggedleft\arraybackslash}p{4.2cm}p{10cm}}

    \textsc{Aug. 2024} & \emph{Organiser of the workshop Profinite Rigidity}, joint with the Universiy of Münster, Regensburg University, and the Karlsruhe Institute of Technology, Karlsruhe, Germany.\\

    \textsc{Apr. 2022} & \emph{Organiser of the RTG "Asymptotic Invariants and Limits of Groups and Spaces"
    Seminar: Group (co)homology} joint with the Karlsruhe Institute of Technology and Heidelberg University, Germany.\\

    \textsc{Oct. 2021 -- Sep. 2022} & \emph{Speaker of the RTG "Asymptotic Invariants and Limits of Groups and Spaces"}.\\

\end{longtable}

%----------------------------------------------------------------------------------------
%	LANGUAGES
%----------------------------------------------------------------------------------------

%\section{Languages}
%
%\begin{tabular}{rl}
%\textsc{Spanish:} & Native\\
%
%\textsc{English:} & Fluent\\
%
%\textsc{German:} & B1\\
%\end{tabular}


\end{document}
